\documentclass{acm_proc_article-sp}
\usepackage[utf8]{inputenc}

\renewcommand{\paragraph}[1]{\vskip 6pt\noindent\textbf{#1 }}
\usepackage{hyperref}
\usepackage{graphicx}
\usepackage{url}

\providecommand{\tightlist}{%
  \setlength{\itemsep}{0pt}\setlength{\parskip}{0pt}}



% Add imagehandling

\numberofauthors{2}
\author{
}

\date{}

%Remove copyright shit
\permission{}
\conferenceinfo{} {}
\CopyrightYear{}
\crdata{}

% Pandoc syntax highlighting

% Pandoc citation processing
\newlength{\csllabelwidth}
\setlength{\csllabelwidth}{3em}
\newlength{\cslhangindent}
\setlength{\cslhangindent}{1.5em}
% for Pandoc 2.8 to 2.10.1
\newenvironment{cslreferences}%
  {}%
  {\par}
% For Pandoc 2.11+
\newenvironment{CSLReferences}[2] % #1 hanging-ident, #2 entry spacing
 {% don't indent paragraphs
  \setlength{\parindent}{0pt}
  % turn on hanging indent if param 1 is 1
  \ifodd #1 \everypar{\setlength{\hangindent}{\cslhangindent}}\ignorespaces\fi
  % set entry spacing
  \ifnum #2 > 0
  \setlength{\parskip}{#2\baselineskip}
  \fi
 }%
 {}
\usepackage{calc} % for calculating minipage widths
\newcommand{\CSLBlock}[1]{#1\hfill\break}
\newcommand{\CSLLeftMargin}[1]{\parbox[t]{\csllabelwidth}{#1}}
\newcommand{\CSLRightInline}[1]{\parbox[t]{\linewidth - \csllabelwidth}{#1}\break}
\newcommand{\CSLIndent}[1]{\hspace{\cslhangindent}#1}


\begin{document}


\begin{longtable}[]{@{}
  >{\raggedright\arraybackslash}p{(\columnwidth - 0\tabcolsep) * \real{0.06}}@{}}
\toprule
\endhead
title: Developing web-based Forensic Investigation App for investigative
analysis using R and Shiny author: - name: Kelly Koh Kia Woon email:
\href{mailto:kelly.koh.2020@mitb.smu.edu.sg}{\nolinkurl{kelly.koh.2020@mitb.smu.edu.sg}}
affiliation: Singapore Management University - name: Manmit Singh email:
\href{mailto:manmits.2020@mitb.smu.edu.sg}{\nolinkurl{manmits.2020@mitb.smu.edu.sg}}
affiliation: Singapore Management University - name: Syed Ahmad Zaki Bin
Syed Sakaf Al-attas email:
\href{mailto:ahmadzaki.2020@mitb.smu.edu.sg}{\nolinkurl{ahmadzaki.2020@mitb.smu.edu.sg}}
affiliation: Singapore Management University \\
abstract: \textbar{} \\
bibliography: sigproc.bib csl: acm-sig-proceedings.csl output:
rticles::acm\_article \\
\bottomrule
\end{longtable}

\hypertarget{introduction}{%
\section{Introduction}\label{introduction}}

Almost every action leaves a digital trail. Major technological shifts
in the past decade have made the collection of digital evidence, such as
GPS records and payment transactions, a significant tool in criminal and
civil investigations(Goodison, Robert, and Brian 2015). It is crucial
that law enforcement agencies could use and transform the ample data
into insightful information to aid their investigations. The growth in
volume and variety of digital data increases the time and resources
needed to analyse them. Furthermore, cases increasingly the analysis of
multiple devices followed by the correlation of the found evidence.

However, most law enforcement personnel lack training in data science.
Without effective data analysis, investigators will struggle to find
relevant information for their cases. And the results of the digital
evidence analysis can be challenged in court when not well-represented.

Given that most of the work in digital forensics involves association to
individuals, law enforcement agencies need forensic tools to approach
identity management. Work in this area will allow for the identification
of likely suspects and other anomalies. By integrating interactive
visualisation with automated analysis techniques, digital data can be
presented in meaningful ways and allow investigators to interactively
guide their investigation.

Addressing this challenge, we designed and developed the Forensic
Investigation App, an open-source interactive application, to allow law
enforcement agents to conduct investigative analysis of GPS and credit
card data. It aims to simplify data exploration and analysis to gain
valuable insights into the behaviour of individuals quickly. The
prototype is built upon VAST Mini Challenge 2's task to find unusual
patterns in GASTech employees' credit card records and GPS tracking
records of their cars. We demonstrate the potential of FIA with five
visualisations to explore data exploration, inferential analysis and
clustering of credit card transactions and GPS data.

This paper documents our approach to design and develop the interactive
application targeted at law enforcement agencies without full time
support of a data science team. This introduction is followed in Section
2 by an explanation of our objectives and motivation. Section 3 provides
a review of existing techniques used to visualize GPS and credit card
data. Section 4 details the data used. Section 5 provides an overview of
the application and its visualisations. Section 6 summaries the findings
from the use case. Section 7 concludes the report and offers ideas for
further development.

\hypertarget{references}{%
\section*{References}\label{references}}
\addcontentsline{toc}{section}{References}

\hypertarget{refs}{}
\begin{CSLReferences}{1}{0}
\leavevmode\hypertarget{ref-goodison2015}{}%
Goodison, Sean E., C. Davis Robert, and A. Jackson Brian. 2015.
{``Digital Evidence and the U.S. Criminal Justice System - Identifying
Technology and Other Needs to More Effectively Acquire and Utilize
Digital Evidence''} 890.
\url{‘https://www.rand.org/pubs/research_reports/RR890.html’}.

\end{CSLReferences}
\setlength{\parindent}{0in}

\end{document}
